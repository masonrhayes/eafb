% Options for packages loaded elsewhere
\PassOptionsToPackage{unicode}{hyperref}
\PassOptionsToPackage{hyphens}{url}
%
\documentclass[
]{article}
\usepackage{lmodern}
\usepackage{amssymb,amsmath}
\usepackage{ifxetex,ifluatex}
\ifnum 0\ifxetex 1\fi\ifluatex 1\fi=0 % if pdftex
  \usepackage[T1]{fontenc}
  \usepackage[utf8]{inputenc}
  \usepackage{textcomp} % provide euro and other symbols
\else % if luatex or xetex
  \usepackage{unicode-math}
  \defaultfontfeatures{Scale=MatchLowercase}
  \defaultfontfeatures[\rmfamily]{Ligatures=TeX,Scale=1}
\fi
% Use upquote if available, for straight quotes in verbatim environments
\IfFileExists{upquote.sty}{\usepackage{upquote}}{}
\IfFileExists{microtype.sty}{% use microtype if available
  \usepackage[]{microtype}
  \UseMicrotypeSet[protrusion]{basicmath} % disable protrusion for tt fonts
}{}
\makeatletter
\@ifundefined{KOMAClassName}{% if non-KOMA class
  \IfFileExists{parskip.sty}{%
    \usepackage{parskip}
  }{% else
    \setlength{\parindent}{0pt}
    \setlength{\parskip}{6pt plus 2pt minus 1pt}}
}{% if KOMA class
  \KOMAoptions{parskip=half}}
\makeatother
\usepackage{xcolor}
\IfFileExists{xurl.sty}{\usepackage{xurl}}{} % add URL line breaks if available
\IfFileExists{bookmark.sty}{\usepackage{bookmark}}{\usepackage{hyperref}}
\hypersetup{
  pdftitle={Problem Set 2},
  pdfauthor={Mason Ross Hayes},
  hidelinks,
  pdfcreator={LaTeX via pandoc}}
\urlstyle{same} % disable monospaced font for URLs
\usepackage[margin=1in]{geometry}
\usepackage{graphicx}
\makeatletter
\def\maxwidth{\ifdim\Gin@nat@width>\linewidth\linewidth\else\Gin@nat@width\fi}
\def\maxheight{\ifdim\Gin@nat@height>\textheight\textheight\else\Gin@nat@height\fi}
\makeatother
% Scale images if necessary, so that they will not overflow the page
% margins by default, and it is still possible to overwrite the defaults
% using explicit options in \includegraphics[width, height, ...]{}
\setkeys{Gin}{width=\maxwidth,height=\maxheight,keepaspectratio}
% Set default figure placement to htbp
\makeatletter
\def\fps@figure{htbp}
\makeatother
\setlength{\emergencystretch}{3em} % prevent overfull lines
\providecommand{\tightlist}{%
  \setlength{\itemsep}{0pt}\setlength{\parskip}{0pt}}
\setcounter{secnumdepth}{-\maxdimen} % remove section numbering
\ifluatex
  \usepackage{selnolig}  % disable illegal ligatures
\fi

\title{Problem Set 2}
\usepackage{etoolbox}
\makeatletter
\providecommand{\subtitle}[1]{% add subtitle to \maketitle
  \apptocmd{\@title}{\par {\large #1 \par}}{}{}
}
\makeatother
\subtitle{for Empirical Analysis of Firm Behavior}
\author{Mason Ross Hayes}
\date{11 November 2021}

\begin{document}
\maketitle

\hypertarget{question-1}{%
\section{Question 1}\label{question-1}}

The distribution of percent misconduct by firm could be described by an
Inverse-\(\chi^2\) distribution or an Inverse-Gamma distribution,
depending on the parameters we choose. Below I show a histogram of 100
random samples from an Inverse-\(\chi^2\) with 14 degrees of freedom to
compare to the firm misconduct histogram. The Inverse-\(\chi^2\) can
have a longer tail, but on the support of firm misconduct, the
histograms are very similar.

\includegraphics{eafb_ps2_files/figure-latex/unnamed-chunk-1-1.pdf}
\includegraphics{eafb_ps2_files/figure-latex/unnamed-chunk-1-2.pdf}

The distribution of percent misconduct at the state level looks more
like a normal distribution. Using the mean and standard deviation of
state percent misconduct and sampling 100 times from a normal
distribution, we get:

\includegraphics{eafb_ps2_files/figure-latex/unnamed-chunk-2-1.pdf}
\includegraphics{eafb_ps2_files/figure-latex/unnamed-chunk-2-2.pdf}

Compared to the firm-level distribution, the state-level distribution
shows less variance; misconduct is much more concentrated at the state
level. This makes sense --- at the firm level we would expect greater
variance, as some firms may have a propensity to engage in misconduct
that influences the (mis)conduct of their employees. Or, firms may be in
some industry where misconduct is more likely for some structural or
organizational reason. Similarly, some firms may be especially strict in
regulating misconduct or may emphasize virtues such as honesty and
integrity in the workplace, leading both to employment self-selection
and to peer effects. These different qualities would vary greatly across
specializations and across firms, but we would expect that the
composition of firms is relatively similar in different states.

At the state-level, where we aggregate across firms that exist in a
given state, it makes sense that the average misconduct will show less
variance between states than between firms. To grossly generalize, the
distribution of aggregated data will always show less variance than
individualized data.

\hypertarget{question-2}{%
\section{Question 2}\label{question-2}}

Before running this regression, there are a couple useful graphs to look
at: what is the distribution of the percent misconduct conditional on
the number of brokers, and what is the distribution of the number of
brokers conditional on the percent misconduct?

\includegraphics{eafb_ps2_files/figure-latex/unnamed-chunk-3-1.pdf}
\includegraphics{eafb_ps2_files/figure-latex/unnamed-chunk-3-2.pdf}

The distribution of the number of brokers for those counties falling in
the bottom quantile of percent misconduct has a longer tail compared to
those in the top quantile: the likelihood of a high number of brokers is
higher. Without this long tail, the distributions of bottom, middle, and
top quantiles all look remarkably similar.

The estimation of state level fixed effects will depend on the quality
of data that we have; for many states, we have only a handful of
observations. This may be insufficient to truly estimate state effects
if we only observe one county --- and in this scenario, we are probably
only observing the most populous or wealthiest counties.

In the following figures I show the distribution of misconduct by state:

\includegraphics{eafb_ps2_files/figure-latex/unnamed-chunk-4-1.pdf}
\includegraphics{eafb_ps2_files/figure-latex/unnamed-chunk-4-2.pdf}
\includegraphics{eafb_ps2_files/figure-latex/unnamed-chunk-4-3.pdf}
\includegraphics{eafb_ps2_files/figure-latex/unnamed-chunk-4-4.pdf}

I have quite a strong prior belief that the distribution of misconduct
is not inherently different across states. However, the fixed effects we
estimate will probably vary greatly, especially since we have few
observations per state. For example, we only observe 3 counties in all
of Oklahoma; this may lead to a biased fixed effect estimate for
Oklahoma.

\begin{table}[!htbp] \centering 
  \caption{Q2 Model: State FE} 
  \label{q2_model} 
\begin{tabular}{@{\extracolsep{5pt}}lc} 
\\[-1.8ex]\hline 
\hline \\[-1.8ex] 
 & \multicolumn{1}{c}{\textit{Dependent variable:}} \\ 
\cline{2-2} 
\\[-1.8ex] & Percent Misconduct \\ 
\hline \\[-1.8ex] 
 nbrokers & $-$0.000003$^{***}$ \\ 
  & (0.000001) \\ 
  & \\ 
 sq\_nbrokers & 0.000000$^{**}$ \\ 
  & (0.000000) \\ 
  & \\ 
\hline \\[-1.8ex] 
Observations & 473 \\ 
R$^{2}$ & 0.349816 \\ 
Adjusted R$^{2}$ & 0.269317 \\ 
Residual Std. Error & 0.027364 (df = 420) \\ 
\hline 
\hline \\[-1.8ex] 
\textit{Note:}  & \multicolumn{1}{r}{$^{*}$p$<$0.1; $^{**}$p$<$0.05; $^{***}$p$<$0.01} \\ 
\end{tabular} 
\end{table}

In Table 1 I show the results of a state-fixed effects regression of
misconduct on the number of brokers and the square of number of brokers,
since we may expect that an increase in the number of brokers has a
diminishing impact. The estimated broker coefficient is statistically
significant but does not seem very meaningful; an increase of 5000
brokers is associated with a 0.0137 lower percent misconduct.

This coefficient is heavily skewed by outliers --- for example, dropping
New York county from the data leads to a 64\% increase in the magnitude
of the estimated coefficient.

\hypertarget{question-3}{%
\section{Question 3}\label{question-3}}

\hypertarget{the-research-idea}{%
\subsection{The Research Idea}\label{the-research-idea}}

Educational attainment may help explain misconduct at the county level.
For example, it may be the case that better education is associated with
greater ethics --- this seems unlikely however, as college education is
an intellectual pursuit and not a character-building endeavor; greater
knowledge does not imply better ethical judgment. There may be some
other factors that are correlated with both education and (mis)conduct;
for example, stronger social ties in the community may be associated
both with greater educational attainment and lower levels of misconduct.
This, however, could be too subjective of a measurement and is in any
case hard to measure.

The more likely and more straightforward relationship between
educational attainment and misconduct is that counties with low
education could see higher misconduct, not because less-educated brokers
are committing misconduct, but rather that less-educated consumers may
have less knowledge of how to report misconduct, less awareness that it
is taking place, or fewer resources to challenge misconduct. Financial
services in counties with a higher level of educational attainment may
also be in wealthier areas; and these could be more competitive,
attracting only the best brokers, with ``lower quality'' brokers
defaulting to more remote locations that may have lower educational
attainment. This is somewhat reminiscent of the claim made by Egan et
al.~2016 that financial advisory firms who engage in misconduct are
``cater(ing) to unsophisticated consumers''.

\hypertarget{the-data}{%
\subsection{The Data}\label{the-data}}

I pull county-level educational attainment data from the Economic
Research Service of the USDA. This includes the number of people at with
a certain level of educational attainment (less than high school, high
school diploma, some college, or bachelor's degree and more), as well as
percent of the population with a certain level of educational
attainment, from 1970 to 2019. It is hard to see why educational
attainment in the current year would be directly related to the
financial misconduct; if it is the case that firms locate in an area of
lower educational attainment to target ``unsophisticated consumers'',
then this location decision would likely be made in advance. Since the
county level data is from 2015, combined with the fact that we do not
have educational data specific to 2015 but rather aggregated from
2015-2019, I choose to use educational data from 2000.

County data for the states of Alabama to Connecticut (in alphabetical
order) is not available after joining by FIPS codes. This is likely
because some counties are included in the original data for which the
FIPS code does not match or does not exist in the educational data.

\hypertarget{regression-results}{%
\subsection{Regression Results}\label{regression-results}}

With this regression, I hope to show if there exists a correlation
between educational attainment and financial misconduct in US counties.
The proposed hypothesis is that counties with lower educational
attainment will show higher levels of misconduct.

The coefficients for number of brokers and for the percent of county
residents with some college education are both significant at the 5\%
level. The effect of the number of brokers seems to be slightly dampened
after adjusting for education levels in the county; it changes from
\(-2.73 \times 10^{-6}\) to \(-1.94 \times 10^{-6}\). For an increase of
one percentage point in those with less than a high school diploma,
there is an decrease of 0.11 percentage points of misconduct; in those
counties with a higher fraction of residents who have lower than a high
school education, there is a higher likelihood of financial misconduct.
This may be misleading however, as areas with low educational attainment
may simply be poorer areas that have lower demand for financial
services; I should add county-level GDP or perhaps county-level poverty
rates to the regression to account for such a possibility.

Furthermore, the estimated coefficients are not robust to the inclusion
of different educational measures; after controlling for \emph{some
college}, for example, the coefficient for \texttt{pct\_less\_than\_hs}
is no longer significant, nor are coefficients of the other educational
variables. Clearly, however, as these educational attainment categories
are mutually exclusive and cover the entire county population, they are
collinear; if a county has a low percentage of population with a
bachelor's degree or higher, it necessarily has a relatively higher
population of people with some college, a high school diploma, or less.
So, we have to have caution in simply adding more variables in the
regression; this is an additional reason that I've chosen to focus only
on those with lower than a high school education and those with higher
than a bachelor's degree.

\begin{table}[!htbp] \centering 
  \caption{Q3 Model: State FE} 
  \label{q3_model} 
\begin{tabular}{@{\extracolsep{5pt}}lc} 
\\[-1.8ex]\hline 
\hline \\[-1.8ex] 
 & \multicolumn{1}{c}{\textit{Dependent variable:}} \\ 
\cline{2-2} 
\\[-1.8ex] & Percent Misconduct \\ 
\hline \\[-1.8ex] 
 nbrokers & $-$0.000002$^{**}$ \\ 
  & (0.000001) \\ 
  & \\ 
 sq\_nbrokers & 0.000000 \\ 
  & (0.000000) \\ 
  & \\ 
 pct\_less\_than\_hs & $-$0.001109$^{***}$ \\ 
  & (0.000402) \\ 
  & \\ 
 pct\_bachelors & $-$0.000427 \\ 
  & (0.000261) \\ 
  & \\ 
\hline \\[-1.8ex] 
Observations & 415 \\ 
R$^{2}$ & 0.354814 \\ 
Adjusted R$^{2}$ & 0.272187 \\ 
Residual Std. Error & 0.027628 (df = 367) \\ 
\hline 
\hline \\[-1.8ex] 
\textit{Note:}  & \multicolumn{1}{r}{$^{*}$p$<$0.1; $^{**}$p$<$0.05; $^{***}$p$<$0.01} \\ 
\end{tabular} 
\end{table}

\newpage

For counties for which we observe the percent with less than high school
education, I display the distributions of county data by state:

\includegraphics{eafb_ps2_files/figure-latex/unnamed-chunk-5-1.pdf}
\includegraphics{eafb_ps2_files/figure-latex/unnamed-chunk-5-2.pdf}
\includegraphics{eafb_ps2_files/figure-latex/unnamed-chunk-5-3.pdf}
\includegraphics{eafb_ps2_files/figure-latex/unnamed-chunk-5-4.pdf}

Additionally, I look at the distribution of percent less than HS
education conditional on where the county falls in number of brokers or
in percent misconduct:

\includegraphics{eafb_ps2_files/figure-latex/unnamed-chunk-6-1.pdf}
\includegraphics{eafb_ps2_files/figure-latex/unnamed-chunk-6-2.pdf}

Notice that, conditional on greater than or equal to 40\% of the
population of a given county having less than a high school education,
it is most likely that that county falls in the top quantile number of
brokers.

\hypertarget{question-4}{%
\section{Question 4}\label{question-4}}

\hypertarget{research-proposal}{%
\subsection{Research Proposal}\label{research-proposal}}

These results are not very illuminating for my initial curiosity of how
does educational attainment interact with financial misconduct. With
this data, there is very little, if any evidence of a statistically
significant or economically meaningful relationship between educational
attainment levels and financial misconduct rates.

While the typical broker would have a Bachelor's degree in finance,
business, economics, or other fields, there is no formal educational
requirement to become a broker. From the current data, there is little
evidence to support the idea that financial misconduct is more frequent
in counties with lower educational attainment. However, we have not
answered the question of whether brokers who have \emph{ever} committed
financial misconduct are more likely to later locate in these areas.

In Question 3, I suggested that perhaps ``lower quality'' brokers may go
to less-educated areas --- this could be a combination of the fact that
areas with lower educational attainment are generally less developed as
well and are home to fewer and less prestigious financial firms. In this
case, a broker's financial misconduct could serve to financial firms as
a signal that that broker is of low quality. If this is the case, we may
expect brokers to begin their career in a highly developed area and/or
at a large financial firm and to then move to a less developed
area/smaller firm after being found guilty of misconduct, since the more
influential financial firms would not be willing to hire them.

It would be insightful to have data on broker educational attainment,
performance (returns), and career trajectories (relocations); the issue
would be collecting this data which could span across multiple counties,
states, and firms.

Every broker in the United States has a Central Registration Depository
(CRD) number; this number may change when they change positions, but is
always tied to their unique ID. By using this identifier, we could track
where particular brokers move over time --- we could see for example if,
after being found guilty of misconduct, a broker is likely to move to a
smaller county, a smaller firm, or a county with lower educational
attainment. Most of this data is already public, but the misconduct data
we have is aggregated. I would need the time, area, firm, and CRD number
associated with the misconduct, and with this data I could attempt to
answer the question: after committing financial misconduct, are brokers
more likely to relocate to an area with a lower level of educational
attainment?

With data on broker education and performance, we could also attempt to
answer the secondary questions: does the distribution of broker
misconduct vary based on brokers' educational attainment ? Are brokers
with a history of low performance more likely to later be found guilty
of misconduct?

\hypertarget{references}{%
\section{References}\label{references}}

Egan, Mark, Gregor Matvos, and Amit Seru. ``The Market for Financial
Adviser Misconduct.'' SSRN Scholarly Paper. Rochester, NY: Social
Science Research Network, September 1, 2017.
\url{https://doi.org/10.2139/ssrn.2739170}.

USDA ERS. ``Educational attainment for the U.S., States, and counties,
1970-2019.'' Accessed November 8, 2021.
\url{https://www.ers.usda.gov/data-products/county-level-data-sets/download-data/}.

\end{document}
